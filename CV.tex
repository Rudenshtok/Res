\documentclass[10pt,A4,english]{article}	
\usepackage[utf8]{inputenc}		
\usepackage[USenglish]{isodate}
\usepackage{fancyhdr}
\usepackage[numbers]{natbib}
\usepackage{xstring, xifthen}
\usepackage{enumitem}
\usepackage[english]{babel}
\usepackage{blindtext}
\usepackage{pdfpages}
\usepackage{changepage}
\usepackage[default]{raleway}
\renewcommand*\familydefault{\sfdefault} 	
\usepackage[T1]{fontenc}
\usepackage{moresize}
\usepackage{fontawesome}
\newcommand{\vcenteredinclude}[1]{\begingroup
\setbox0=\hbox{\includegraphics{#1}}%
\parbox{\wd0}{\box0}\endgroup}
\newcommand{\tab}[1]{\hspace{.2\textwidth}\rlap{#1}}
\newcommand*{\vcenteredhbox}[1]{\begingroup
\setbox0=\hbox{#1}\parbox{\wd0}{\box0}\endgroup}
\newcommand{\icon}[3] { 							
	\makebox(#2, #2){\textcolor{maincol}{\csname fa#1\endcsname}}
}	
\newcommand{\icontext}[4]{ 						
	\vcenteredhbox{\icon{#1}{#2}{#3}}  \hspace{2pt}  \parbox{0.9\mpwidth}{\textcolor{#4}{#3}}
}
\newcommand{\iconhref}[5]{ 						
    \vcenteredhbox{\icon{#1}{#2}{#5}}  \hspace{2pt} \href{#4}{\textcolor{#5}{#3}}
}
\newcommand{\iconemail}[5]{ 						
    \vcenteredhbox{\icon{#1}{#2}{#5}}  \hspace{2pt} \href{mailto:#4}{\textcolor{#5}{#3}}
}
\usepackage{paracol}
\usepackage{tikzpagenodes}
\usetikzlibrary{calc}
\usepackage{lmodern}
\usepackage{multicol}
\usepackage{lipsum}
\usepackage{atbegshi}
\usepackage[a4paper]{geometry}
\geometry{top=1cm, bottom=1cm, left=1cm, right=1cm}
\usepackage{fancyhdr}
\pagestyle{empty}
\setlength{\parindent}{0mm}
\usepackage{array}
\newcolumntype{x}[1]{%
>{\raggedleft\hspace{0pt}}p{#1}}%

\usepackage{graphicx}
\usepackage{tikz}			
\usepackage{ragged2e}	
\usetikzlibrary{shapes, backgrounds,mindmap, trees}

\usepackage{transparent}
\usepackage{color}

\definecolor{maincol}{RGB}{ 64,64,64}
\definecolor{darkcol}{RGB}{ 70, 70, 70 }
\definecolor{lightcol}{RGB}{245,245,245}
\definecolor{accentcol}{RGB}{59,77,97}
\usepackage[hidelinks]{hyperref}
\newcommand{\mpwidth}{\linewidth-\fboxsep-\fboxsep}

\newcommand{\cvlist}[1] {
	\begin{itemize}{#1}\end{itemize}
}

\newcommand{\cvtext}[1] {
	\begin{tabular*}{1\mpwidth}{p{0.98\mpwidth}}
		\parbox{1\mpwidth}{#1}
	\end{tabular*}
}
\newcommand{\cvtextsmall}[1] {
	\begin{tabular*}{0.8\mpwidth}{p{0.8\mpwidth}}
		\parbox{0.8\mpwidth}{#1}
	\end{tabular*}
}
\newcommand{\cvsection}[1] {
	\vspace{14pt}
	\cvtext{
		\textbf{\LARGE{\textcolor{darkcol}{#1}}}\\[-4pt]
		\textcolor{accentcol}{ \rule{0.2\textwidth}{1.5pt} } \\
	}
}

\newcommand{\cvsectionsmall}[1] {
	\vspace{14pt}
	\cvtext{
		\textbf{\Large{\textcolor{darkcol}{#1}}}\\[-4pt]
		\textcolor{accentcol}{ \rule{0.2\textwidth}{1.5pt} } \\
	}
}

\newcommand{\cvheadline}[1] {
	\vspace{16pt}
	\cvtext{
		\textbf{\Huge{\textcolor{accentcol}{#1}}}\\[-4pt]
		 
	}
}

\newcommand{\cvsubheadline}[1] {
	\vspace{16pt}
	\cvtext{
		\textbf{\huge{\textcolor{darkcol}{#1}}}\\[-4pt]
		 
	}
}
\newcommand{\cvskill}[3] {
	\begin{tabular*}{1\mpwidth}{p{0.72\mpwidth}  r}
 		\textcolor{black}{\textbf{#1}} & \textcolor{maincol}{#2}\\
	\end{tabular*}%
	
	\hspace{4pt}
	\begin{tikzpicture}[scale=1,rounded corners=2pt,very thin]
		\fill [lightcol] (0,0) rectangle (1\mpwidth, 0.15);
		\fill [accentcol] (0,0) rectangle (#3\mpwidth, 0.15);
  	\end{tikzpicture}%
}

\newcommand{\cvevent}[7] {
	
	\parbox{\mpwidth}{
		\begin{tabular*}{1\mpwidth}{p{0.66\mpwidth}  r}
	 		\textcolor{black}{\textbf{#2}} & \colorbox{accentcol}{\makebox[0.32\mpwidth]{\textcolor{white}{\textbf{#1}}}} \\
			\textcolor{maincol}{#3} & \\
		\end{tabular*}\\[8pt]
	
		\ifthenelse{\isempty{#4}}{}{
			\cvtext{#4}\\
		}
	}
	\vspace{14pt}
}

\newcommand{\cvmetaevent}[4] {
	\textcolor{maincol} { \cvtext{\textbf{\begin{flushleft}#1\end{flushleft}}}}

	\ifthenelse{\isempty{#2}}{}{
	\textcolor{black} {\cvtext{\textbf{#2}} }
	}

	\ifthenelse{\isempty{#3}}{}{
		\cvtext{{ \textcolor{maincol} {#3} }}\\
	}

	\cvtext{#4}\\[14pt]
}

\newcommand{\cvqrcode}[1] {
	\begin{center}
		\includegraphics[width={#1}\mpwidth]{qrcode}
	\end{center}
}

\newcommand\Header[1]{%
\begin{tikzpicture}[remember picture,overlay]
\fill[accentcol]
  (current page.north west) -- (current page.north east) --
  ([yshift=50pt]current page.north east|-current page text area.north east) --
  ([yshift=50pt,xshift=-3cm]current page.north|-current page text area.north) --
  ([yshift=10pt,xshift=-5cm]current page.north|-current page text area.north) --
  ([yshift=10pt]current page.north west|-current page text area.north west) -- cycle;
\node[font=\sffamily\bfseries\color{white},anchor=west,
  xshift=0.7cm,yshift=-0.32cm] at (current page.north west)
  {\fontsize{12}{12}\selectfont {#1}};
\end{tikzpicture}%
}

\newcommand\Footer[1]{%
\begin{tikzpicture}[remember picture,overlay]
\fill[lightcol]
  (current page.south east) -- (current page.south west) --
  ([yshift=-80pt]current page.south east|-current page text area.south east) --
  ([yshift=-80pt,xshift=-6cm]current page.south|-current page text area.south) --
  ([xshift=-2.5cm,yshift=-10pt]current page.south|-current page text area.south) --	
  ([yshift=-10pt]current page.south east|-current page text area.south east) -- cycle;
\node[yshift=0.32cm,xshift=9cm] at (current page.south) {\fontsize{10}{10}\selectfont \textbf{\thepage}};
\end{tikzpicture}%
}

%%%%%%%%%%%%%%%%%%%%%%%%5
\begin{document}

\columnratio{0.31}
\setlength{\columnsep}{2.2em}
\setlength{\columnseprule}{4pt}
\colseprulecolor{white}


% LEBENSLAUF HIERE
\AtBeginShipoutFirst{\Header{CV}\Footer{1}}
\AtBeginShipout{\AtBeginShipoutAddToBox{\Header{CV}\Footer{2}}}

\newpage

\colseprulecolor{lightcol}
\columnratio{0.31}
\setlength{\columnsep}{2.2em}
\setlength{\columnseprule}{4pt}
\begin{paracol}{2}


\begin{leftcolumn}

\includegraphics[width=\linewidth]{resources/Me.jpg}
	\fcolorbox{white}{white}{\begin{minipage}[c][1.5cm][c]{1\mpwidth}
		\LARGE{\textbf{\textcolor{maincol}{Rudenko Varvara}}} \\[2pt]
		\normalsize{ \textcolor{maincol} {Student of MIPT} }
\end{minipage}} \\


\cvsection{Skills}

\cvskill{Data science} {1+ yrs.} {0.70} \\ [-2pt]

\cvskill{Data visualisation} {4+ yrs.} {0.75} \\ [-2pt]

\cvskill{Machine learning} {1+ yrs.} {0.70} \\[-2pt] \\

\cvskill{SQL} {1+ yrs.} {0.64} \\[-2pt]

\cvskill{Python} {2+ yrs.} {0.64} \\[-2pt]

\cvskill{C++} {2+ yrs.} {0.20} \\[-2pt]

\cvskill{C} {2+ yrs.} {0.16} \\[-2pt]

\cvskill{LaTeX} {2+ yrs.} {1} \\[-2pt]\\

\cvskill{English} {B1} {0.64} \\[-2pt]
\cvskill{French} {A2} {0.20} \\[-2pt]

\newpage

\cvsection{Education}

\cvmetaevent
{09/2019 - 06/2021}
{Department of Radio Engineering and Cybernetics}
{Moscow Institute of Physics and Technology}

\cvmetaevent
{06/2021 - today}
{Department of Control and Applied Mathematics}
{Moscow Institute of Physics and Technology}

\cvsection{Interests}

\icontext{CaretRight}{12}{Singing}{black}\\[6pt]
\icontext{CaretRight}{12}{Gymnastics}{black}\\[6pt]
\icontext{CaretRight}{12}{Cooking}{black}\\[6pt]
\icontext{CaretRight}{12}{Travel}{black}\\[4pt]




\cvsection{Contact}

\icontext{MapMarker}{16}{Dolgoprudny}{black}\\[6pt]
\icontext{MobilePhone}{16}{+79251564897}{black}\\[6pt]
\iconemail{Envelope}{16}{Rudenko.VD@phystech.edu}{Rudenko.VD@phystech.edu}{black}\\[6pt]
\iconhref{Github}{16}{github.com/Rudenshtok}{https://github.com/Rudenshtok}{black}\\[6pt]

	
\end{leftcolumn}
\begin{rightcolumn}


\cvsection{Biography}
\vspace{4pt}

Hello, in this resume I want to introduce myself and show the reasons why you can choose me for further work and training. I know how to learn and am willing to spend time developing myself in the areas that interest me. I hope that after graduating from the university, the acquired "knowledge base" will help me get not only a well-paid job, but also an interesting one)))

\vspace{10pt}
\cvsection{Work experience}
\vspace{4pt}

\cvevent
{11/2021 - today}
	{Trainee Researcher}
	{International Laboratory of Stochastic Algorithms and Multidimensional Data Analysis}
	{Training, understanding and optimization of artificial intelligence models. Work on articles on the topic of RL.}
	\vfill\null


\vspace{10pt}
\cvsection{Additional education}
\vspace{4pt}


\cvevent
{08/2020}
	{Student}
	{Sirius University of Science and Technology in the program "Modern methods of information theory, optimization and management" with the direction " Sampling, management and optimization"}
	{An educational trip in which the following materials were studied: Monte Carlo methods, lectures on optimization and control theory, and an introduction to reinforcement learning.A personal project has been prepared, under the guidance of Alexey Naumov.}
	\vfill\null



\cvevent
{2021}
	{Student}
	{"SQL for Data Science", University of California, Davis}
	{The course on the theoretical component of the programming language was attended, the classical set of functions necessary for elementary queries in Data Science was studied.}
	\vfill\null


	
\cvevent
    {2021}
	{Student}
	{"Fundamentals of Reinforcement Learning", The Alberta Institute ${\&}$ Alberta Machine Intelligence Institute}
	{The basic theory in the field of Machine learning is studied, with a deepening in reinforcement learning and an analysis of the main tasks and problems. The problem of the "multi-armed bandit" is studied in depth outside the course.}
	\vfill\null
	\vfill\null


   	
\cvevent
{03/2021 - 05/2021}
	{Student}
	{"Data Analysis in the Industry", Tinkoff}
	{Introduction to the work of data analysts in the company, a conversation with employees. Attended a course of lectures on Data analysis: A B-tests, Logistic regression, Random forest, and so on. Tasks designed to help you understand the topic and deal with the tasks that occur in the routine work of the analyst were completed.}
	\vfill\null
	
\cvevent
{07/2021 - 08/2021}
	{Student}
	{Sirius University of Science and Technology in the program "Modern methods of information theory, optimization and management" with the direction "Stochastic algorithms and machine learning"}
	{An educational trip with an already familiar laboratory, where work on the article was carried out under the guidance of Alexey Naumov. The team conducted research and proved in practice new approaches to working with UVIP for RL.}
	\vfill\null





\cvsection{Projects}

\begin{itemize}[leftmargin=*]
    \item Importance Sampling and control variates\\
    There exist many problems in science and engineering whose exact solution
either does not exist or is difficult to find. For the solutions of those
problems, one has to resort to approximate methods. The Variational
Monte Carlo (VMC) technique is relatively insensitive to the size of the
system, it can be applied to large systems where some other methods are
computationally not feasible.
    \item Gaussian Process Optimization in the Bandit Setting:No
Regret and Experimental Design\\
    Analysis of the article and work on it: Many applications require optimizing an unknown, noisy function that is   expensive to evaluate. We formalize  this  task  as  a  multi-armed bandit problem, where the payoff function is either sampled from a Gaussian process (GP) or has low RKHS norm. We resolve the important open problem of deriving regret bounds for this setting, which imply novel convergence rates for  GP  optimization. We  analyze GP-UCB, an intuitive upper-confidence based algorithm, and bound its cumulative regret in terms of maximal information gain, establishing a novel connection between GP optimization and experimental design.
	\item UVIP: Model-Free Approach to Evaluate Reinforcement Learning Algorithms\\
    During the shift, the task assigned to me was completed, as well as, due to the early completion of the work, helping a person on another part of the project. It was proposed to use the KBSF method to estimate the probabilistic transition and an algorithm was written that simplifies the work, unlike the classical KBRL. After that, I began to help a person with the implementation of the UVIP algorithm itself on the Atari games platform, we got quite good results, which we need to work on in the future.
\end{itemize}

\cvsection{Recommendation}
    \newline
    \includegraphics[width=\linewidth]{resources/3.png}

\end{rightcolumn}
\end{paracol}


\end{document}